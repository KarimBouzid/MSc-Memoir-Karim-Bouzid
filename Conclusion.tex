The portable impedance-sensing device described in this memoir, coupled with the microfluidics systems, are effectively capable of measuring and estimating the properties of the microparticles of sizes going as low as 50 $\mu$m. The system created for this MSc is novel, and the first found in the scientific literature to achieve such a sensitivity level while boasting a small size, low-cost, and low power-consumption. The impedance-sensing device takes 50 mm $\times$ 50 mm $\times$ 15 mm of space, while the microfluidics system takes 46 mm $\times$ 25 mm $\times$ 50 mm. The impedance-sensing system needs 1.5 W to function adequately, and is powered by a battery of a voltage between 2.5 V and 3 V. The impedance can be sampled for frequencies between 30 kHz and 15 MHz. The impedance spectroscopy of discrete resistors and complex discrete models were obtained and their impedance spectrum closely followed the predictions. The impedance profiles of saline solutions were also obtained, from which microbeads were added. The impedance of the solution with microbeads was analyzed in the microfluidic system and 2238 microbeads were detected. From that dataset, the important parameters of microbeads were found, which proved the usefulness of the biosensor as a portable device capable of measuring micron-scale particles. \par

The project could be improved in a couple of ways. (1) The electronic of the impedance-sensing system uses mostly discrete components such as op-amps to accomplish functions which would be better suited to integrated electronics. A custom-made IC would be perfect for this applications and could probably reduce the power-consumptions from 1.25 W down to a couple of mW. (2) Using a different 3D-printer, such as the one designed by \citep{Gong2017}, which achieved structures with minimal microchannel widths of 20 $\mu$m, would help increase the sensitivity of the particle detection. (3) The electronics of the impedance-sensing system and PCB fabrication could be improved to reduce the baseline noise values. (4) The ADCs ranges could be tuned to better fit the measured results, which would increase the ENOB in return.