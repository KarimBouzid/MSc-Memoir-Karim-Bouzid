Microfluidic channels can be fabricated using different techniques. The one producing the best resolution (around 2µm), which is also the most popular in research, is photolithography in cleanrooms \cite{Olanrewaju2018}. Direct patterning of the channels on silicon substrates can also be done using deep reactive ion etching. Soft lithography can also be chosen to reduce the mask price, at the cost of a decrease in resolution (around 10µm for plastic masks). A mold can be created and PDMS can be casted on that mold to create a channel using only an oven and simple consumable \cite{Olanrewaju2018}. A final sealing layer of glass or even PDMS can be used to seal tight the channel and the mold can be reused. This technique unfortunately requires expensive cleanroom facilities, complex fabrication processes, and high fabrication times \cite{Olanrewaju2018}. PDMS replication is adequate for academic settings (approximately 4h) but is too slow and uneconomic for industrial processed, which can achieve channels in minutes using hot embossing, or even in a couple of seconds using injection molding. 3d-printing can also be used to create molds in which PDMS can be cured. Multiple 3d-printing currently exist, but the one providing the best results for high resolutions and low surface roughness is stereolithography-based 3d-printing \cite{Olanrewaju2018,Gong2017}. UV light is applied to a resin basin, which incrementally moves in the z-axis. The resin is cured in layers and replicates a 3d-model designed on CAD software. The smallest feature produced using this technique is around 90µm, despite the resolution being claimed by manufacturer to be around 30µm. This can be explained because of the resin used these applications that tend to spread during curing, meaning that they are not especially suitable for microfluidics application \cite{Olanrewaju2018,Gong2017}. Recently, \citep{Gong2017} designed a homemade 3d-printer specifically for microfluidics which can attain truly microscopic scales of 18 x 20 µm by modifying the type of resin used and optimizing the stereolithographic process. 