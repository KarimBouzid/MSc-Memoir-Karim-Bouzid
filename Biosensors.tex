A biosensor is defined as “an analytical device, used for the detection of a chemical substance, that combines a biological component with a physicochemical detector” \cite{turner1987biosensors}. A biosensor is a self-contained apparatus that interacts with a specific bioreceptor (e.g. enzyme, antibody, DNA, proteins, etc.) taken from a given analyte (e.g. tissues, cells, polluted water, soil, etc.) using a physicochemical transducer. The transducer translates the electrochemical changes taking place in the analyte that modify the parameters of the bioreceptor into a useful signal (generally electrical or optical) that can be sampled by electronical circuits \cite{Kim2019}. Biosensors can be used to monitor a broad range of biorecognition events or be engineered to target specific ones. The transducer can be of an electrochemical (amperometric, potentiometric, impedimetric, and conductometric), optical, electrical, gravimetrical, pyroelectrical, or piezoelectrical nature. Current, voltage, resistance, capacitance, permittivity, light scattering, mass, and temperature, among others, can all be measured as a function of time - or as a frequency spectrum - to monitor bio-electrochemical reactions. The specificity and sensitivity of the biosensor depends on the detection mechanism, the way the bioreceptor is interacted with, and the type, shape, and size of the transducer \cite{Maas2018}. \par

Unlike laboratory-based biosensors, those designed for portable applications can be affected by harsh and fluctuating environmental conditions for prolonged duration in unsupervised environments. These biosensors can consequently be affected by shortcomings such as a gradual biofouling at the analyte-sensor interface, a limited stability of numerous bioreceptors, changing environmental conditions which can denature the obtained results, and an inefficient interaction between the sample and the sensor. The biofouling is characterized by an accumulation of proteins, cells, or macromolecules through unspecific binding on the sensor’s surfaces. This adsorption hampers the target analyte’s diffusion to the sensor’s surface and gradually decreases the sensitivity and performance over time. The variation in environmental conditions (such as temperature, pressure, humidity, pH, etc.) modifies the regions of operation of the device. An active calibration is required to consider these parameters, which is time and resource consuming. The interfaces that constitute these biosensors are paramount to the performances of the device, considering the heterogeneity of biological elements \cite{Kim2019}. Tissues, for example, are constituted of different cellular layers with differing biochemical markers; soils are constituted from an aggregate of inert molecules, particles, microorganisms, etc.; water is a biofluid in which flows minerals, inert particles, microplastics, microorganisms, etc. The specificity of the biosensor should thus be sufficient to measure the desired analyte in place of the other components, or alternative solutions should be erected to compensate this lack. \par

The first general design constraints \cite{horowitz1989art} defined at this point of the study concerning the fabrication of the biosensor to study microparticles are:

\begin{itemize}
	\item Portability. The system is targeted for uses in situ directly in the fields.
	\item Online and offline monitoring. The system should monitor and trace important parameters in real-time when needed and store them in memory for further offline post-processing.
	\item Simplicity. Complexity in electronic systems is usually associated with an increase in power-consumption and noise magnitude. Keeping it as simple as possible will allow the system better performances and ease of use. 
	\item Affordability. Ideally, the system will be widely used to gather macroscopical data about microorganisms unknown to date. A low-price is thus required for such a large scall implementation. 
	\item Robustness. It should be possible to use the system in harsh conditions of temperature and humidity, in the presence of contaminants, dusts and dirt, with little to no modifications needed for the bio-sample, and for lengthy periods of time.
\end{itemize}