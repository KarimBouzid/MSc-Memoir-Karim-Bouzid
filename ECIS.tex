EIS can be applied to cells bounded to a substrate to obtain better results and specificity. Electric Cell-Substrate Impedance Sensing (ECIS) uses EIS on cells adsorbed on the device’s electrodes in order to follow the cell’s activities in real-time such as the variations in shape, spreading area and tightness of adherent cells. The cell’s membrane acts as an insulator compared to the ion-rich microbial broth such that the impedance measured from the electrodes increases as the confluence of cells on the electrode is established. Confluent cells adhered to the electrode’s surface effectively reduce the electrodes’ surface area, thus offering an impedance that varies only accordingly to the cell’s shape and tightness \cite{Wegener1999,Xu2016}. At low frequency, the electrical field created by the electrode has an easier time passing through the cell’s junction, whereas at high frequency, the cell’s cytoplasm becomes short-circuited, which makes it a lower impedance path for the electrical field. These properties can finally be monitored by measuring the impedance frequency spectrum as a function of time and following its localized changes: a low frequency change corresponds to a variation in cell-to-cell and cell-to-electrode tightness; and a high frequency change corresponds to a variation in the cell’s volume or shape \cite{AppliedBiophysics}. \par

The downside of this technique is that in a heterogeneous analyte, some compounds other than the targeted microorganism can adhere to the electrode and falsify or at least denature the obtained results. To counter this, the use of antibody to increase the SUT’s specificity is prominent in ECIS. These antibodies are immobilized on the sensor’s surfaces to functionalize it for a specific biological sensing application. This action also decreases the value of the critical microbial concentration threshold $C_{TH}$, which makes it possible to detect even lower concentration of bacteria and decreases the overall Detection Time (DT) \cite{Lei2014}. The use of antibodies to increase the specificity makes ECIS a poor choice for an application that aims to identify unknown microorganisms of varying properties and sizes, despite that it is a non-invasive and label-free technique. Another shortcoming of this technique is the poor stability of the immobilized cells and the great sensitivity of the performance with the choice of the immobilization technique \cite{Grossi2017}. \par

ECIS nevertheless possesses the advantage of permitting the analysis of microorganisms on solid culture media. The SUT is dropped and let to diffuse on a solid medium in contact with the electrodes so that the microbes may adhere to them. Despite its simpler fabrication and better portability, ECIS for solid media culturation provides similar performance to liquid media \cite{Choi2009}.
