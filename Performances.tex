The performances and characteristics of the impedance-sensing device and microfluidics system are summarized in \autoref{tab:Performances}.

\begin{table}[h]
\centering
\caption{\label{tab:Performances} Measured system performances}
\begin{tabular}{|c | c|} 
 \hline
 Parameter & Value \\ [0.5ex] 
 \hline
 Supply voltage (battery) & 2.5 V to 3 V  \\ 
 \hline
 Maximum power consumption & 1.75 W \\
 \hline
 Nominal power consumption & 1.25 W \\
 \hline
 Input voltage @ electrodes & 450 mV \\
 \hline
 Sampling frequency & 5461 Sps or 650 Sps \\ 
 \hline
 Excitation frequency range & 30 kHz to 15 MHz \\
 \hline
 Excitation frequency resolution & 100 Hz \\
 \hline
 Type of excitation signal & Square \\ 
 \hline
 Impedance magnitude range & 800 $\Omega$ to 50 k$\Omega$ \\
 \hline
 Impedance magnitude precision & < 3\% \\
 \hline
 Impedance phase precision & < 3\% \\
 \hline
 PCB size & 50 $\times$ 50 $\times$ 15 mm \\ [1ex] 
 \hline
 Microchannel size & 150 $\times$ 150 $\mu$m \\
 \hline
 Electrode width & 106 $\mu$m \\
 \hline
 Electrode separation & 424 $\mu$m \\
 \hline
\end{tabular}
\end{table}

The power consumption is adequate for a portable application. It could, however, be greatly reduced by transforming the PCB to an integrated application in a custom-made IC, considering that the vast majority of the power is dissipated in the op-amps, which serve to do only basic functions such as inverting and amplifying signals. This is something that will be explored in the next phases of the project. 