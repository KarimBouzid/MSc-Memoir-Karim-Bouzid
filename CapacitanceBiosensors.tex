This review of capacitance biosensors was features in Seyedeh Nazila Hosseini's "A Comprehensive Review of Advances in Electrochemical Biosensors for Microbial Monitoring" that will be published in a couple of months in 2022 for TBioCAS. \par

Capacitance-based biosensors comprise such measurement method built upon a measure of the capacitance of an analyte. Such techniques can be found in the scientific literature in a variety of forms. The most important ones concerning bacterial monitoring are: Charge Sharing capacitive sensors, Charge-Based capacitance measurement (CBCM) and Capacitance-to-Frequency Converter, which are all high resolution and compact solutions to measure capacitances. \par

The first technique relies upon the charge sharing principle. Two capacitors $C_{N1}$ and $C_{N2}$ with known values of capacitances are placed between DC tension nodes $V_{DD}$ and $V_{SS}$ and separated by three MOSFET switches M1, M2 and M3. This switched capacitance configuration creates two intermediate nodes N1 and N2 associated with respective nodal tension $V_{N1}$ and $V_{N2}$. The unknown capacitance is then placed at one of the nodes and the measurement is accomplished by two successive operations realized with the MOSFETs. Firstly, during what is called the reset phase, the switches M1 and M3 are turned on while M2 is kept blocking. This charges the capacitances at nodes N1 and N2 to $V_{DD}$ and $V_{SS}$ respectively. Then, during the evaluation phase, the switches M1 and M3 are shut while M2 is activated, which causes a redistribution of the charges accumulated in $C_{N1}$ and $C_{N2}$ that leads to a nodal tension which follows:
\begin{equation}
   V_{N}(t) = \frac{V_{DD} (C_{N1}+C_{sensed}) + C_{N2} V_{SS}}{C_{N1}+C_{N2}+C_{sensed}}
\end{equation}
The only unknown parameter becomes the sensed capacitance, which can then be deduced easily [1]:
\begin{equation}
   C_{sensed}(t) = \frac{(V_{DD}-V_{SS}) C_{N2} - (C_{N1}+C_{N2}) V_N}{V_N}
\end{equation}
This configuration can be integrated efficiently on sensing electrode to minimize die space and features a great linear high sensitivity dependence between the output voltage and the sensed capacitance. Compensation schemes for the Fixed-Noise Pattern (FNP) are also available in the literature to further increase the performances [2]. However, the use of switching capacitors is not free from drawbacks. Indeed, such circuits suffer from speed limitation caused by the time needed to fully charge and discharge the capacitances. They also necessitate two non-overlapping clock signals in order to function properly, otherwise charge injection and clock feed-through caused by the MOSFET switches would cause gain error, offset and distortion \cite{horowitz1989art}[3]. The inverse relation between $C_{sensed}$ and $V_N$ creates a limited range of measurable capacitance with an adequate precision [1]. \par

The second technique is Charge-Based Capacitance Measurement (CBCM) which permits, in its original version proposed in 1996 by J. C. Chen [4] a really high precision and sensitivity attaining attofarad resolution. The simplest version of this circuit is made of two switching MOSFETs M1 and M2 with a sensed capacitance Csensed in between them, both placed between the power supplies Vss and Vdd.  They are both excited by non-overlapping excitation signals, that successively charge and discharge the sensing capacitance. The non-overlapping signal is required to prevent short-circuiting the power-supplies. Supposing that the switches M1 and M2 are ideals and that the excitation frequency of the non-overlapping clock signals is low enough to allow a full charge and discharge of the sensed capacitance, the transient current $I_C$ in the switches becomes proportional to the frequency of the clock signal $f$, the reference tension $V_{DD}$ and the sensed capacitance $C_{sensed}$. In this situation, the form of the transient current is of no importance, only the average current value is used to obtain the sensed capacitance value [4]. The average current furnished by the power-supply then becomes the sum of the quiescent current and the transient current:
\begin{equation}
   I_{AVG} = \frac{I_{DC}}{2} + C_{sensed} V_{step} f
\end{equation}
By adding an integrator at $V_{DD}$ or $V_{SS}$ made with a capacitor of value $C$, it becomes possible to deduce the average current, which further allows the deduction of the sensed capacitance value [7][10]: 
\begin{equation}
   V_{out} = -V_{DD} - \frac{T}{C} I_{AVG}
\end{equation}
It has since then been significantly improved upon and applied notably for particles detection and bacteria monitoring. One of the first upgrades made upon this simple circuit was to add a CMOS current mirror at the power supply in order to replicate the current waveform of the charge and discharge of the sensed capacitance in another branch of the circuit, which could then be used for referential purposes for a differential measurement. Another improvement that can be realized on this circuit is to measure the average current for different voltage value of the power-supply $V_{DD}$: the slope of the best linear regression through the data obtained can be used to calculate a better sensed capacitance less affected by the parasitic capacitances present in the model [10]. On-board compensation schemes can also be added to directly eliminate the parasitics, at the cost of additional on-board die, power-consumption, and complexity \cite{horowitz1989art}[8]. \par

CBCM are effective capacitance sensors since they do not require elaborate circuits in order to attain adequate results. They do not, in comparison to voltammetry, amperometry and electrochemical impedance spectroscopy (EIS), require readout circuits or signal conditioning modules, which makes them incredibly compact and suitable for high-throughput applications where a low on-chip area is required [3]. The resolution limit is effectively caused by the mismatch between the drain junction and overlap capacitance of the transistors of the design [4]. The charging time of the capacitances can become extremely small for highly conductive electrolytes, which makes the sensitivity of this device the main point to optimize [2]. \par

The final technique uses a Current-to-Frequency Converter (CFC) to change a capacitance value to a frequency, which timings can be measured by any digital circuit that can count reference crossings. This approach generally combines a Current-To-Voltage Converter (CVC) and a Voltage-to-Frequency Converter (VFC) into one compact design to bypass the need for intermediate signal constituents. One such design can be accomplished with a simple differential design: a constant current is sent to a sensing capacitance and a reference electrode; the difference is calculated and the resulting waveform is integrated until a reference threshold is attained, at which point an output is produced and the input is reset; the output frequency thus depends on the capacitance difference between the electrodes [5]. Another way to accomplish a CFC is by devising a self-oscillating circuit in a closed-loop pattern based on an integrating amplifier and a comparator with set reference values, with an input current that oscillates based on its output. This way, the tension at the electrode is successively increased and decreased proportionally to its capacitance value, and a bit-stream at the comparator’s output is created that has a frequency dependent on the rate of change of the electrode:
\begin{equation}
   \frac{1}{f} = 2 R C * ln(\frac{1}{1-\frac{V_{ref}}{I_{ref} R}})
\end{equation}
Where $R$ and $C$ are the electrode-electrolyte interfacial parameters, $I_{ref}$ is the input current and $V_{ref}$ is the magnitude value of the voltage references for the comparator [6]. \par

The main advantage of CFC sensors is that they eliminate the need for an ADC circuit, which makes them cost and space-efficient [6]. They however suffer from a high FNP when miniaturized because no compensation schemes exist for them [11]. Additionally, the need for a comparator makes this configuration less optimal for highly integrated designs and high throughput application, such as Multi-Electrode Array (MEA) for spatial recognition of particles [2].\par

The main concern about capacitive measurement applied to bacteria measurement is that the sensed capacitances are generally extremely small, typically a few femtofarads, which means that the parasitic capacitances and noise cannot be neglected. Additionally, the capacitance conversion can be further troubled by the high conductivity of the bacterial medium. To achieve better precision and to reduce the effects of parasitic, a differential configuration is often used. This, however, is far from optimal for high-throughput application because each sensed capacitance must be linked to a similar valued reference capacitance, which reduces the versatility of the sensor for a broader range of applications considering that the fixed calibration resistance is calibrated for a given electrolyte [2]. Capacitance sensors also do not suffer from measurement drifts caused by long-time monitoring, which counters the need for a self-calibration module [9].\par