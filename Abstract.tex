To this day, a couple of biosensors have been proposed to quickly and easily measure the features and properties of individual microrganisms members of an heterogeneous population, but none of these approaches were adequate candidates to perform measurements directly in the field. Biosensors for micron-scale organisms generally require extreme sensitivity or specificity, which are difficult to combine with a portable general device. This study proposes a device based on Impedance Flow-Cytometry that can detect and quantify the size and velocity of microbeads of size bigger than 50 µm, while boasting a low cost, low size, low power, and simplicity of design and operation utilizing the potential of 3D-printing and industrial PCB fabrication. An example is provided for a Big Data application from a sampled dataset containing 2380 detected microbeads of sizes between 50 µm and 90 µm. 

\etdFrontMatter{Résumé}
À ce jour, quelques biocapteurs ont été proposés pour mesurer rapidement et facilement les caractéristiques et les propriétés des microrganismes individuels membres d'une population hétérogène, mais aucune de ces approches ne s'est avérée être adéquate pour effectuer des mesures directement sur le terrain. Les biocapteurs pour les organismes microscopiques nécessitent généralement une sensibilité ou une spécificité extrême, qui sont difficiles à combiner avec un dispositif général portatif. Cette étude propose un dispositif basé sur la cytométrie de flux d'impédance qui peut détecter et quantifier le diamètre de microbilles de tailles supérieure à 50 µm, tout en présentant un faible coût, une taille réduite, une basse consommation de puissance et une simplicité de conception et d'opération qui utilise le potentiel de l'impression 3D et de la fabrication industrielle de circuits imprimés. Un exemple est offert afin de démontrer les capacités du capteurs pour de larges échantillons, avec un jeu de données contenant 2380 microbilles détectées de tailles entre 50 µm et 90 µm. 